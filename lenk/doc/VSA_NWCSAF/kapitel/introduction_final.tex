\chapter{Introduction}
This document comprises the report of the work carried out by TROPOS scientists within the framework of the M\'{e}t\'{e}o France -- EUMETSAT associate scientist activity for the verification of the NWC\,SAF Convective Initiation (CI) product developed for the v2018 software release. 

The NWC\,SAF CI product is currently being developed and improved at the Nowcasting Departement of M\'{e}t\'{e}o France in Toulouse. While it is currently an demonstrational product, it is expected to become a tool for operational use by forecasters for the early detection of potentially hazardous convective events. As stated in the user manual and the detailed theoretical base algorithm description of the first version of the product \citep{Autones2016a, Autones2016b}, the product is based on principles of CI detection algorithms described previously in the literature.  Specifically, these are the SATCAST algorithm \citep{MecikalskiBedka2006, MecikalskiBedkaPaechEtAl2008, SieglaffCronceFeltzEtAl2011}, which was developed at the University of Huntsville (Alabama, USA), and the  CbTRAM algorithm \citep{ZinnerMannsteinTafferner2008, MerkZinner2013} developed at the German Aerospace Center. 

A first version of the CI product was developed during the NWC\,SAF project phase CDOP2 and was released as part of the version 2016 GEO software package. As stated in the validation study for this product version by \citet{Karagiannidis2016}, the product shows the potential to become a valuable tool for forecasters, but exhibits a significant rate of false alarms similar to the SATCAST algorithm. In the verification study, \citet{Karagiannidis2016} also points out several starting points for improvement of the product. These points include the use of the NWC\,SAF cloud type to only select areas where cumuliform clouds are present, and to mask out cirrus clouds which are also known to cause erroneous detections of convective initiation by the SATCAST algorithm. Another point is to use the NWC\,SAF cloud microphysics product (CMIC) to select regions that are indeed susceptible for convective initation. The incooperation of additional interest fields from numerical weather prediction models such as stability indices or low level convergence is also deemed promising. \citet{Karagiannidis2016} also suggests to use information from the solar SEVIRI channels, to improve the tracking in order to enable a longer forecast horizon, and to retrain the thresholds used by the current and future versions of the algorithm based on logistic regression.

As stated in a presentation held by Jean-Marc Moisselin \citep{Mousselin2017} at the DWD CM\,SAF seminar in Offenbach (Germany) in October 2017, a number of suggestions by \citet{Karagiannidis2016} have been incorporated into v2018 of the NWC\,SAF CI product. The thresholds and the thresholding approach have been adapted, the tracking has been improved, and eligible pixels are selected based on the cloud type and the cloud microphysical products. 

Based on the new version of 2018 of the CI product, this study targets an objective validation of  the detection capability of this improved version based on radar data, and tries to quantify the improvements in comparison to the previous version 2016. For these goals, the RADOLAN composite from Germany's network of weather radars is used as ground truth.

This report is structured as follows: in Chapters 2 and 3, the datasets and methods used for our evaluation are described. Together with a field-based analysis presented in Chapter 4, an attempt toward an object-based validation is made in Chapter 5, including a discussion of the benefits and problems arising from an object-based approach. Conclusions and recommendations are given in Chapter 6.


%\Blindtext
