\chapter{Conclusion and Recommendations}

This report summarises the investigations carried out in the framework of this NWC\,SAF associate scientist activity, which was targeted at the validation of the v2018 CI product. 

A number of methodological improvements have been made over the previous validation, which include the following: 1) using motion fields to forecast the future location of CI events and thus enable a more meaningful validation \add[HD]{with radar observations used by us as ground truth in this study}; 2) parallax-correction of the satellite-based products to achieve better overlap with the radar observations used as ground truth; 3) a first attempt towards an object-based evaluation of the product. 

In the first result chapter (Chapter 4), several field-based analyses have been presented. First, the diurnal cycles of the total areal coverage of the CI product have been contrasted with the areal coverage of the raining area obtained from ground-based radar. \change[FS]{This comparison reveals rather different behaviours for the CI product's low and high probability classes (based on probabilities of $>50\%$ and $\leq50\%$,respectively): the low-probability classes cover a significant areal fraction of the validation domain for the case days and show little correspondence to the diurnal cycle of the areal coverage of rain areas obtained from radar, which suggests a too-high sensitivity and poor discriminative skill for convective development, while a much smaller areal coverage and a direct correspondence of the high-probability classes to subsequent increases in precipitating area have been found for the high-probability classes.}
{The comparison reveals a rather different behaviour of the CI product at different probability levels. The low-probability classes with a probability less than 50\% cover a significant areal fraction of the validation domain for the case days and show little correspondence to the diurnal cycle of the areal coverage of rain areas obtained from radar. This suggests a too-high sensitivity and poor discriminative skill for convective development of the low-probability classes.
In contrast, a much smaller areal coverage and a direct correspondence of the high-probability classes to subsequent increases in precipitating area have been found for the high-probability classes.} 
Two additional consistency aspects of the CI product have been investigated: CI warnings should not include regions with existing significant precipitation, and due to the definition of CI, temporal persistence of warnings should generally not exceed the forecast horizon of the CI product. Both aspects were found to be satisfactory for the high-probability levels \add[HD]{, while deficiencies were noted for the lower-probability levels}.

\add[SL]{In the second result chapter (Chapter 5), the object-based analyses and the product validation statistics have been presented. A total number of 1180 single cell cloud objects with a minimum life time of 30\,min were derived for the five case days. This number corresponds to 526 radar-based ground truth objects. Almost all of the ground truth objects have at least one corresponding cloud object. However, this relatively large number of objects does not translate into robust validation results: most of the cloud objects correspond to true negative cases, and only 15 cases remain as true positives. Only for two CI probability classes, 25--50\% and 75--100\%, validation results can be calculated. These are a POD of 0.52, a FAR of 0.76 , a CSI of 0.19 and a HSS of 0.31 for the 25--50\% class and a POD of 0.28, a FAR of 0.50, a CSI of 0.21 and a HSS of 0.34 for the 75--100\% class. While the small number of cases limits the statistical significance of these results, the scores indicate a reduced false alarm rate compared to the previous version of the CI product.}
\add[HD]{These numbers suggest that the typically rather high FAR has been reduced in the v2018 product at the cost of a relatively low POD, which indicates that further tuning of the product to achieve and optimal trade-off between POD and FAR might be needed.}

\add[SL]{Considering the differing number of cloud and ground truth objects for the case days, it should be noted that the contribution of the individual case days to the overall statistics is not equal. There also seems to be a slight tendency towards an increased number of false negatives on some of the case days, which could be an indication for a geographical or weather pattern dependency of the product. However, as the number of case days is quite small, a larger data base should be considered to support such conclusions.
}

Based on the results of this study, a number of recommendations can be made: 

1. From our results and the skill-scores presented in Chapter 5, we believe that the currently under-development v2018 CI product constitutes a clear improvement over its predecessor v2016, in particular considering the fact that our validation using ground-based radar fields as truth is rather strict. Note that this statement is only directed at the high-probability CI forecasts ($>50\%$), while the fraction of false positives for the low-probability CI levels ($\leq50\%$) remains rather high. A clear improvement seems to result from the use of cloud-type information. At this stage, our validation statistics suggest a FAR and POD of about 50\% and 30\%, respectively, which are close to target and threshold accuracies given in the product requirements. Considering the relatively new development status of the CI product, we think that it can meet the product requirements and become a valuable tool for forecasters in the future, after some further tuning of thresholds and possibly other improvements. A central limitation to this conclusion is however the fact that the current amount of test cases used for our study is much too small for obtaining robust statistics and \change[HD]{identify}{for identifying} specific weaknesses of the product. Hence, the validation should be repeated using significantly more case days. In particular, day-to-day differences in the object properties and numbers suggest that the product skill might depend \remove[HD]{strongly} on the synoptic conditions. Hence, there is a strong probability that the current selection of case days can bias our validation results. A strategy for optimising both thresholds  and validation scores simultaneously (e.g. using logistic regression together with k-fold cross-validation) seems the most promising avenue for this goal and is recommended here.

2. A critical aspect for both the computation and the validation of the CI product is an accurate treatment of cloud motion, a point already raised in the previous validation study by \citet{Karagiannidis2016}. Here, sensitivities arise through the calculation of time trends, and for obtaining validation scores by overlapping the CI product with radar fields observed at later times. In our present study, we have applied the TV-L1 motion estimation algorithm to track the satellite-based cloud and CI fields, and to forecast their future motion. A more thorough investigation of the accuracy of the underlying motion fields and their impact for the calculation of time trends, the overall impact on the accuracy of the CI product, and for the resulting validation scores seems highly desirable. It also seems noteworthy that consistency between the motion fields used for the calculation of time trends as part of the CI algorithm, and for forecasting the future location of CI events seems a goal.  

3. Based on forecaster feedback, we believe it is desirable to transform the current field-based \change[JMM]{Ci}{CI} product at least optionally into an object-based output format. This suggestion is also motivated by the notion that true CI events should correspond to the development of discrete convective cells. An object-based CI product necessitates the combination of multiple connected CI forecast pixels into single objects, including strategies for merging different-probability pixels and calculating the total object CI probability. The approach chosen for our object-based validation presented in Chapter 5 can be seen as a first attempt towards this goal, but requires further refinement due to the complexity \add[HD]{of this task}. In particular, the steps of identification and tracking of cloud objects, the handling of splits and merges of these objects, and the subsequent assignment of cloud objects to radar-based ground truth objects for validation require parameter choices\change[HD]{ and all}{, which} introduce individual uncertainties. At this stage, due to the limited amount of test data, it remains unclear whether the benefits of an object-based CI product or validation outweigh the complexity of this approach, and how sensitive our analyses are to some ad-hoc choices made for the individual steps. 
