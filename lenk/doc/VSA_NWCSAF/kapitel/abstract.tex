\renewcommand\abstractname{Executive summary}
\begin{abstract}
\addcontentsline{toc}{chapter}{Executive summary}
This report summarises the results of the analyses which have been conducted in the framework of the M\'{e}t\'{e}o France -- EUMETSAT associate scientist activity to validate the improved Convection Initiation (CI) product developed for the NWC\,SAF v2018 GEO software package.  

The detection of clouds that will develop into thunderstorms is an important but challenging task in operational weather forecast. Hence, a number of different satellite-based approaches to aid weather forecasters in forecasting thunderstorms have been developed within the last two decades. However, currently only few are used operationally. The demonstrational NWC\,SAF v2016 Convection Initiation product has been a valuable first attempt to deliver an operationally applicable convection initiation detection algorithm for the European geostationary weather satellites. Despite having a high amount of false alarms, the product was considered to be useful for forecasters in a previous validation study.

For the 2018 version of the product, a number of changes have been made to improve the forecasting accuracy and in particular to reduce the high amount of false alarms. This validation study is aimed at the evaluation of the forecasting performance of the new product version to verify these changes.

The validation has been conducted using both a pixel-based and an object-based approach. For the object-based approach, the CI probabilities determined on a per-pixel basis by the product have been assigned to cloud objects derived from the high resolution visible channel of the MSG observations, and have been validated using precipitation objects derived from radar data of the German weather radar network as ground truth. 

The validation results confirm that the amount of false alarms indeed has been reduced compared to the last version of the product at a value of about 50\%. The probability of detection remains relatively low at 26\%. Unfortunately, the limited number of case days yields only a small set of objects for validation, limiting the statistical significance of these validation results. 

Concerning future improvements of the product, a number of possible aspects are recommended for consideration. First, a transformation of the currently pixel-based into an object-based product seems promising, as it enhances the usability for forecasters and can also facilitate the tracking of potential thunderstorm objects over time. Secondly, a re-assessment of the atmospheric motion fields and their influence on the overall detection capability is proposed. Lastly, a retraining of the algorithm thresholds is recommended, possibly taking a dependency on synoptic conditions into account. 
\end{abstract}
